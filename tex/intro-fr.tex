%*******************************************************************************
%*******************************************************************************
\addentry{toc}{section}{\textbf{Introduction}}
\chapter*{Introduction}
\minitoc
\label{chap:introduction}
\chaptermark{Introduction}
%*******************************************************************************
%*******************************************************************************

Dans ce document, nous expliquons comment utiliser et tirer pleinement
partie des outils et resources proposés sur \url{http://www.patacrep.com} pour
réaliser des recueils de chansons. Tout d'abord, une présentation
rapide des deux différents projets~:

\paragraph{Songbook}
Ce projet correspond à un ensemble de scripts facilitant la production
de recueils de chansons. Il contient également l'ensemble des
tablatures des chansons. Vous pouvez par exemple télécharger le pdf
complet sur \url{http://www.patacrep.com/data/documents/songbook.pdf}
pour vous faire une idée du rendu final.

\paragraph{Songbook-client} 
Il s'agit d'une interface graphique au songbook facilitant la création
de recueils personnalisés.

\paragraph{Technologies utilisées} 
Le songbook réutilise le projet \emph{Songs LaTeX
  Package}\footnote{\url{http://songs.sourceforge.net/}} pour le rendu
des chansons.

\begin{itemize}
\item \LaTeX~: les chansons sont écrites en respectant une syntaxe
  particulière qui sera décrite ultérieurement. Cette contrainte
  permet d'utiliser \LaTeX pour le rendu du texte avec tous les
  avantages qui en découlent~;
\item Lilypond~: outil utilisé pour l'écriture des partitions~;
\item Makefile~: le pdf est généré automatiquement à partir des sources. Tout le processus
  de compilation est automatisé par makefile. Ainsi, la production d'un pdf s'effectue
  via la commande \emph{make} dans un terminal.
\item Python~: certains éléments comme les index par titre/auteur
  sont générés avec Python.
\item C++/Qt4~: langage/toolkit retenu pour la réalisation du songbook-client.
\end{itemize}

\paragraph{Licences}
Tout le code est distribué sous licence GPLv2
(\url{http://www.gnu.org/licenses/gpl.html}). Tous les documents et
autres resources sont distribués sous une licence Creative Commons
CC-By-Sa\footnote{\bysa~\url{http://creativecommons.org/}}.
