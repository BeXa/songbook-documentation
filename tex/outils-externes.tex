%*******************************************************************************
%*******************************************************************************
\chapter{Outils externes}
\setcounter{chapter}{3}
\label{chap:outils-externes}
\minitoc
%*******************************************************************************
%*******************************************************************************

%*******************************************************************************
\section{\LaTeX{}}
%*******************************************************************************

\LaTeX{} est un logiciel de traitement de texte qui va générer un
document pdf depuis un fichier source (un simple fichier texte). Cela
permet à l'utilisateur de se concentrer sur le contenu plutôt que sur
la forme. Pour que \LaTeX{} puisse produire un rendu correct, la
rédaction d'un fichier source doit respecter certaines contraintes
basées sur l'utilisation de commandes.

Une commande commence par un antislash ($\backslash$) suivi du nom de
la commande. Ensuite viennent les arguments, s'il y en a, entre
accolades (obligatoires) ou crochets (optionnels).


%*******************************************************************************
\section{Git}
%*******************************************************************************

%commandes de base
%git clone add/rm commit push/pull status

%récupérer un dépôt

%mettre à jour

%modifier

%command avancées
%git branch amend checkout rebase
%github

%*******************************************************************************
\section{Lilypond}
%*******************************************************************************

La documentation du projet Lilypond est très
claire et se trouve sur le site \url{http://lilypond.org/}.
Voici néanmoins quelques concepts de base~:

\begin{itemize}
\item les lettres a, b, c, d, e, f, g représente les notes la, si, do,
  ré, mi, fa, sol~;
\item un chiffre derrière une lettre en indique la durée (2=blanche, 4=noire,
  8=croche) et un point après un chiffre désigne une note pointée~;
\item \emph{ais}, \emph{bes} désignent un \emph{la dièse} et un \emph{si bémol}
\item \emph{'} et \emph{,} servent à monter/descendre d'une octave.
\end{itemize}

Lilypond est généralement empaqueté pour les distributions
GNU/Linux. Pour des distributions basées Debian/Ubuntu, l'installation
se fait par~:

\begin{unixcom}
  sudo apt-get install lilypond
\end{unixcom}
