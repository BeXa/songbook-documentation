%*******************************************************************************
%*******************************************************************************
\addentry{toc}{section}{\textbf{Introduction}}
\chapter*{Introduction}
\minitoc
\label{chap:introduction}
\chaptermark{Introduction}
%*******************************************************************************
%*******************************************************************************

Dans ce document, nous expliquons comment utiliser et tirer pleinement
partie des outils proposés sur Patacrep! pour réaliser des carnets de
chansons. Tout d'abord, une présentation rapide des différents
projets~:

\paragraph{Songbook}
La base du projet en lui-même. Il s'agit d'un ensemble de chansons
écrites en respectant quelques conventions (simples) et des outils qui
permettent de produire le pdf correctement mis en forme. Vous pouvez
par exemple télécharger le pdf complet sur
\url{http://www.patacrep.com/data/documents/songbook.pdf} pour
vous faire une idée du rendu final.

\paragraph{Songbook-client} 
Il s'agit d'une interface graphique permettant essentiellement de
sélectionner facilement une sous-liste de chansons afin de produire un
recueil personnalisé.

\paragraph{Technologies} 
Différents langages et outils sont utilisés par ces projets. Nous
utilisons particulièrement le projet \emph{Songs LaTeX
Package}\footnote{\url{http://songs.sourceforge.net/}} pour le rendu
des chansons.

\begin{itemize}
\item \LaTeX~: les chansons sont écrites en respectant une syntaxe
  particulière qui sera décrite ultérieurement. Cette contrainte
  permet d'utiliser \LaTeX pour le rendu du texte avec tous les
  avantages qui en découlent~;
\item Lilypond~: outil utilisé pour l'écriture des partitions~;
\item Makefile~: le mécanisme de compilation des fichiers est
  automatisé par makefile.  La génération d'un recueil ne demande donc
  qu'un simple ``make'' dans un terminal.
\item Python~: certains éléments comme les indexes par titre/auteur
  sont générés avec Python.
\item C++/Qt4~: langage retenu pour la réalisation du songbook-client.
\end{itemize}

\paragraph{Licences}
Tout le code est distribué sous licence GPLv2
(\url{http://www.gnu.org/licenses/gpl.html}). Tous les documents et
autres resources sont distribués sous une licence Creative Commons
CC-By-Sa (\url{http://creativecommons.org/}).
